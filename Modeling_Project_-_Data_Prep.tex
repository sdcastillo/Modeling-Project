\documentclass[]{article}
\usepackage{lmodern}
\usepackage{amssymb,amsmath}
\usepackage{ifxetex,ifluatex}
\usepackage{fixltx2e} % provides \textsubscript
\ifnum 0\ifxetex 1\fi\ifluatex 1\fi=0 % if pdftex
  \usepackage[T1]{fontenc}
  \usepackage[utf8]{inputenc}
\else % if luatex or xelatex
  \ifxetex
    \usepackage{mathspec}
  \else
    \usepackage{fontspec}
  \fi
  \defaultfontfeatures{Ligatures=TeX,Scale=MatchLowercase}
\fi
% use upquote if available, for straight quotes in verbatim environments
\IfFileExists{upquote.sty}{\usepackage{upquote}}{}
% use microtype if available
\IfFileExists{microtype.sty}{%
\usepackage{microtype}
\UseMicrotypeSet[protrusion]{basicmath} % disable protrusion for tt fonts
}{}
\usepackage[margin=1in]{geometry}
\usepackage{hyperref}
\hypersetup{unicode=true,
            pdftitle={Milliman Modeling Project - Data Prep},
            pdfauthor={Sam Castillo},
            pdfborder={0 0 0},
            breaklinks=true}
\urlstyle{same}  % don't use monospace font for urls
\usepackage{graphicx,grffile}
\makeatletter
\def\maxwidth{\ifdim\Gin@nat@width>\linewidth\linewidth\else\Gin@nat@width\fi}
\def\maxheight{\ifdim\Gin@nat@height>\textheight\textheight\else\Gin@nat@height\fi}
\makeatother
% Scale images if necessary, so that they will not overflow the page
% margins by default, and it is still possible to overwrite the defaults
% using explicit options in \includegraphics[width, height, ...]{}
\setkeys{Gin}{width=\maxwidth,height=\maxheight,keepaspectratio}
\IfFileExists{parskip.sty}{%
\usepackage{parskip}
}{% else
\setlength{\parindent}{0pt}
\setlength{\parskip}{6pt plus 2pt minus 1pt}
}
\setlength{\emergencystretch}{3em}  % prevent overfull lines
\providecommand{\tightlist}{%
  \setlength{\itemsep}{0pt}\setlength{\parskip}{0pt}}
\setcounter{secnumdepth}{0}
% Redefines (sub)paragraphs to behave more like sections
\ifx\paragraph\undefined\else
\let\oldparagraph\paragraph
\renewcommand{\paragraph}[1]{\oldparagraph{#1}\mbox{}}
\fi
\ifx\subparagraph\undefined\else
\let\oldsubparagraph\subparagraph
\renewcommand{\subparagraph}[1]{\oldsubparagraph{#1}\mbox{}}
\fi

%%% Use protect on footnotes to avoid problems with footnotes in titles
\let\rmarkdownfootnote\footnote%
\def\footnote{\protect\rmarkdownfootnote}

%%% Change title format to be more compact
\usepackage{titling}

% Create subtitle command for use in maketitle
\newcommand{\subtitle}[1]{
  \posttitle{
    \begin{center}\large#1\end{center}
    }
}

\setlength{\droptitle}{-2em}

  \title{Modeling Project - Data Prep}
    \pretitle{\vspace{\droptitle}\centering\huge}
  \posttitle{\par}
    \author{Sam Castillo}
    \preauthor{\centering\large\emph}
  \postauthor{\par}
      \predate{\centering\large\emph}
  \postdate{\par}
    \date{29 November 2018}


\begin{document}
\maketitle

{
\setcounter{tocdepth}{2}
\tableofcontents
}
\section{Introduction}\label{introduction}

This is a modeling exercise project similar to a kaggle competition. The
goal is to build a predictive model.

Blah blah {[}see @doe99, pp. 33-35; also @smith04, ch. 1{]}.

\subsection{Data}\label{data}

\begin{itemize}
\tightlist
\item
  train.txt -- training dataset
\item
  test.txt -- holdout dataset
\item
  sample.txt -- sample of the format for evaluation
\item
  The training set includes a field labeled ``Amount''
\item
  The holdout set does not include the ``Amount'' field as we would like
  you to prepare predictions for this file
\item
  Each file contains an ``ID'' field that is unique to each observation
  to be predicted
\end{itemize}

\subsection{Objective}\label{objective}

• Build a predictive model that predicts the ``Amount'' variable. • The
model will be evaluated on the mean absolute error (MAE) between the
predicted target and the actual target for the holdout dataset. Some
tradeoff between model accuracy and simplicity/interpretability is
acceptable as long you can justify your modeling decisions. • Determine
what variables are most important for making your model predictions. •
Prepare a .csv or .txt file containing your prediction for each ``ID''
in the same format as the sample provided in sample.txt. • Feel free to
use whatever tools available to you to approach the problem.

Why use MAE instead of root mean square error (RMSE)? MAE is the mean
absolute value difference between the predictions, \(\hat{y_i}\), and
the target, \(y_i\). This means that positive errors are penalized in
the same way as negative errors.

\[\text{MAE} = \frac{1}{n}\sum{|y_i - \hat{y_i}|}\] Root mean squared
error treats positive and negative errors equally, just like MAE, but
imposes a harsher penalty outliers, observations where there is a large
error. This is because it takes the square of the error.

\[\text{RMSE} = \sqrt{\frac{1}{n}\sum{(y_i - \hat{y_i})^2}}\] The
takeaway is that outliers will be less of an issue in this analysis than
in RMSE were used.


\end{document}
